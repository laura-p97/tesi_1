\chapter{Conclusioni}

In questa tesi sono stati analizzati i dati raccolti da ALICE in collisioni protone-protone ad LHC all'energia del centro di massa di 5~TeV e si è studiata la produzione del mesone $D^{*+}$ contenente un quark charm attraverso la ricostruzione del decadimento $D^{*+} \rightarrow D^{0}(\rightarrow \pi^+ K^-)\pi^+$. L'analisi è stata svolta utilizzando un algoritmo di analisi multivariata, il Boosted Decision Tree.

Il lavoro di questa tesi ha mostrato che \`e possibile utilizzare questi algoritmi per l'identificazione del mesone $D^{*+}$ e che l'utilizzo di questa tecnica risulta pi\`u utile quando si considerano candidate con un basso impulso trasverso ($p_T < 5 $~GeV/c) in quanto in questo caso il rapporto segnale su fondo \`e minore di 1. Al contrario, i metodi di analisi multivariata non sono particolarmente utili ad alto impulso trasverso delle candidate, in quanto in questo caso le selezioni topologiche permettono comunque l'identificazione del mesone $D^{*+}$. Infatti, dal paragone dei risultati ottenuti applicando l'analisi multivariata con l'algoritmo del BDT con quelli ottenuti selezionando i mesoni $D^{*+}$ con selezioni sulle variabili topologiche si osserva che con il metodo usato in questa tesi la significanza e l'efficienza per $p_T <$ 5 GeV/c sono pi\`u alte di quelle ottenute con il metodo delle selezioni topologiche. Per valori di $p_T >$ 5 GeV/c il metodo basato sulla selezione di variabili topologiche risulta pi\`u vantaggioso.

I risultati ottenuti con l'utilizzo del BDT potrebbero essere ulteriormente migliorati: uno studio sistematico del valore del taglio della variabile BDT response potrebbe ulteriormente aumentare la significanza e l'efficienza di selezione del mesone $D^{*+}$; sarebbe interessante estendere la misura ad intervalli di $p_T < 2$~GeV/c che non e' stato considerato in questa tesi ma in cui ci si aspetta delle buone performance dell'analisi multivariata; un campione di dati per il training pi\`u grande permetterebbe di estendere l'intervallo di impulso trasverso in cui \`e stata eseguita l'analisi.

Considerando i risultati ottenuti in questa tesi sar\`a interessante in futuro applicare lo stesso metodo per lo studio della produzione del mesone $D^{*+}$ in collisioni piombo-piombo in cui il rapporto segnale su fondo a basso $p_T$ \`e ancora peggiore che nel caso di collisioni protone-protone.  

