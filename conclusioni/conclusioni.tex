\chapter{Conclusioni}

In questa tesi sono stati analizzati i dati raccolti da ALICE in collisioni protone-protone ad LHC all'energia del centro di massa di 5~TeV e si è studiata la produzione del mesone $D^{*+}$ contenente un quark charm attraverso la ricostruzione del decadimento $D^{*+} \rightarrow D^{0}(\rightarrow \pi^+ K^-)\pi^+$. L'analisi è stata svolta utilizzando un algoritmo di analisi multivariata, il Boosted Decision Tree.
\\Il lavoro di questa tesi ha mostrato che \`e possibile utilizzare questi algoritmi per l'identificazione del mesone $D^{*+}$ e che l'utilizzo di questa tecnica risulta pi\`u utile quando si considera un basso impulso trasverso, caso in cui il rapporto segnale su fondo \`e minore di 1, piuttosto che alto impulso trasverso, dove semplici selezioni topologiche permettono comunque l'identificazione del mesone $D^{*+}$.
\\I risultati ottenuti con l'algoritmo del BDT sono stati paragonati con quelli ottenuti selezionando i mesoni $D^{*+}$ con selezioni sulle variabili topologiche. In particolare, si osserva che con il metodo usato in questa tesi la significanza e l'efficienza sono pi\`u alte di quelle ottenute con il metodo delle selezioni topologiche per $p_T <$ 5 GeV/c confermando che il metodo multivariato risulta particolarmente utile a basso impulso trasverso.
\\I risultati ottenuti con l'utilizzo del BDT potrebbero essere migliorati avendo a disposizione un campione di dati pi\`u grande per la fase di training, questo permetterebbe anche di estendere l'intervallo di impulso trasverso in cui \`e stata eseguita l'analisi. Inoltre uno studio più sistematico del valore del taglio della variabile BDT response potrebbe portare dei risultati migliori in quanto a significanza e sezione d'urto del mesone $D^{*+}$.