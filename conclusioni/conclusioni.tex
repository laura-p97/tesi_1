\chapter{Conclusioni}

In questa tesi sono stati analizzati i dati raccolti da ALICE relativi alla collisione di fasci di protoni ad LHC. In particolare si è studiata la produzione del mesone con charm $D^{*+}$. L'analisi è stata svolta utilizzando un algoritmo di analisi multivariata, il Boosted Decision Tree.
\\La fase di training è la prima nell'implementazione del BDT. Per il training del segnale del BDT sono stati utilizzati dati ottenuti da simulazione Monte Carlo. Per il training del fondo non sono stati utilizzati dati da simulazione Monte Carlo, ma si è scelto di utilizzare una parte dei dati raccolti da ALICE di cui si era sicuri ci fosse solo fondo. Questa scelta è stata presa in quanto i risultati dell'analisi sono migliori utilizzando i dati di ALICE per il training del fondo. Ciò accade perché i dati ottenuti da simulazione Monte Carlo non sono del tutto aderenti a quelli di ALICE che si vogliono analizzare. Questo suggerisce che se i dati da simulazione Monte Carlo fossero più simili a quelli reali, il training del segnale potrebbe fornire risultati migliori. 
\\L'algoritmo del BDT ha combinato assieme le variabili discriminatorie che gli erano state fornite, restituendo un'unica variabile chiamata BDT response, che ha permesso di differenziare il segnale dal fondo. Nello scegliere i valori di taglio del BDT response si sono considerate più opzioni, scegliendo infine quella che restituiva i risultati migliori nella fase di applicazione del BDT.
\\Al termine dell'analisi si è confrontata la significatività ottenuta dall'analisi svolta in questa tesi con quella dell'analisi standard di ALICE. Si è visto che per i valori di impulso trasverso più bassi $p_T \ < 4 GeV/c$ la significatività ottenuta utilizzando il BDT è più alta. Mentre al crescere del $p_T$ la significatività dell'analisi standard di ALICE supera quella ottenuta con l'analisi svolta in questa tesi. Questo avviene principalmente per due motivi:
    \begin{itemize}
        \item il fatto che i dati per il training utilizzati diminuiscano all'aumentare del $p_T$, rendendo così più difficoltoso il training del BDT;
        \item la maggiore facilità nel riconoscere il picco del segnale al crescere del $p_T$, che rende più facile l'analisi standard di ALICE.
    \end{itemize}
    
\\Il lavoro di questa tesi ha quindi mostrato che si possono utilizzare algoritmi di analisi multivariata, in particolare Boosted Decision Tree, per analizzare i dati di produzione del mesone $D^{*+}$. Si è evidenziato, inoltre, che i risultati del BDT sono migliori per intervalli di $p_T$ più basso. 
\\I risultati dell'analisi del BDT potrebbero essere migliorati avendo a disposizione per il training una quantità di dati più grande e dei dati che siano più aderenti a quelli raccolti dall'esperimento ALICE. Inoltre uno studio più accurato del valore del taglio della variabile BDT response potrebbe portare dei risultati migliori. Non è stato possibile mettere in atto uno studio di questo tipo in quanto al di fuori della possibilità del lavoro di questa tesi. 