\chapter{ALICE ad LHC}

\section{LHC}
A poca distanza da Ginevra si trova il più grande e il più potente collisore di particelle del mondo: il \textit{Large Hadron Collider} (LHC). Ha un raggio complessivo di 27 Km ed è stato costruito a partire dal 1998 dall'Organizzazione Europea per la Ricerca Nucleare (CERN)con lo scopo di avanzare nella conoscenza dell'Universo e dei suoi meccanismi più profondi. Gli esperimenti condotti dal CERN hanno permesso di avere evidenze sperimentali delle principali teorie che spiegano come funziona la materia di cui l'Universo è composto, un esempio è la scoperta del bosone di Higgs. 
\\In figura Fig~\ref{fig:CERNcomplex} è mostrato il complesso di acceleratori del CERN, di cui LHC rappresenta l'ultimo stadio. I vari acceleratori del CERN permettono al fascio di particelle finali di avere energie e caratteristiche adeguate agli esperimenti che vengono condotti. È composto da un iniziale acceleratore lineare (Linac2), seguono tre sincrotroni, il Proton Synchrotron Booster (PSB), il Proton Synchrotron (PS) e il Super Proton Synchrotron (SPS) dal quale si ottengono particelle accelerate a 450 $GeV$, che vengono infine iniettate nel LHC dove arrivano ad un'energia di 7 $TeV$. \cite{tesi_barbano}
 
    \begin{figure}[htbp]
        \centering
        \includegraphics[width=0.8\linewidth]{ALICE/CernComplex_2018.png}   
        \caption{Complesso dell'intero acceleratore del CERN \\\small{Si vedono  \textcolor{blue}{LHC \textit{Large Hadron Collider}} \textcolor{cyan}{SPS \textit{Super Proton Synchrotron}} \textcolor{purple}{ PS \textit{Proton Synchrotron}} \textcolor{violet}{BOOSTER \textit{ Proton Synchrotron Booster}} LINAC \textit{Linear ACcelerator}} \\{\footnotesize  \textcolor{red}{AD \textit{Antiprotron Decelerator}} \textcolor{green}{ISOLDE \textit{Isotope Separator Online DEvice}}  \textcolor{lightgray}{LEIR \textit{Low Energy Ion Ring}} }}
        \label{fig:CERNcomplex}
    \end{figure}
    
Il fascio di particelle viaggia in un tubo in cui viene fatto l'ultra-vuoto ed è direzionato da dei magneti superconduttivi che devono essere tenuti alla temperatura di 1.85 $K$.
\\Dal 2008 sono operativi i quattro principali esperimenti di LHC: ALICE, ATLAS, CMS e LHCb. Questi hanno preso dati sia in
collisioni protone-protone (\textit{pp}) sin in collisioni piombo-piombo (\textit{Pb-Pb}) ed altre. In questa tesi si utilizzano dati derivanti dall'esperimento ALICE che verrà  descritto più nel dettaglio nel seguito. \cite{sito_cern_LHC}


\section{ALICE}

L'acronimo ALICE sta per \textit{"A Large Ion Collider Experiment"} e il suo scopo è quello di studiare la materia ad alte densità e alte temperature in uno stato chiamato \textit{quark-gluon plasma} (QGP). ALICE è ottimizzato per lo studio dei prodotti da collisioni ad alte energie di ioni pesanti, casi in cui è possibile studiare il QGP mentre si espande e raffredda. L'obbiettivo è quello di riuscire a comprendere lo stato della materia nei primi microsecondi dopo il Big Bang e in quale modo dal QGP si siano create le particelle di cui è composto ora l'Universo. \cite{sito_cern_ALICE} 
    
    \begin{figure}[htbp]
        \centering
        \includegraphics[width=0.6\linewidth]{ALICE/ALICE_LRsaba_CERN_0212_3219.jpg}
        \caption{Foto del rivelatore ALICE}
        \label{fig:ALICEcomplex}
    \end{figure}
    
ALICE è a sua volta composto da vari rivelatori di particelle che permettono di tracciare le traiettorie delle particelle prodotte nella collisione dei fasci, identificarle e misurare varie quantità come la velocità o il momento. Di seguito si riportano alcuni dettagli sui principali detector di ALICE, i cui dati sono stati utilizzati ai fini di questa tesi.

    \subsection{Inner Tracking System (ITS)} \label{ITS}
    L'\textit{Inner Tracking System} è un rivelatore composto da sei cilindri concentrici di rivelatori in silicio di raggio minimo 3.9 $cm$ e massimo 43.0 $cm$, posizionati con l'asse parallelo alla direzione del fascio. Viene utilizzato per:
    \begin{itemize}
        \item la determinazione della posizione del vertice primario. Quest'ultimo è il punto in cui decade una particella generata nella collisione dei fasci, i cui prodotti possono poi decadere a loro volta (in questo caso il punto di decadimento è chiamato vertice secondario). %è una definizione esatta del vertice primario?
        \item la ricostruzione dei vertici secondari di decadimenti di particelle che decadono per interazione debole. 
        \item migliorare la precisione nella determinazione di angolo e momento dati dalla Time Projection Chamber (TPC) (di cui si parla in seguito al \ref{TPC})
        \item permetter il tracciamento di particelle che non vengono tracciate dalla TPC a causa delle sue limitazioni sulla misura del momento delle particelle
        \item identificazione di particelle (PID) con basso momento trasverso sfruttando la perdita specifica di energia $dE/dx$
    \end{itemize}
    
    I rivelatori di ITS hanno una risoluzione spaziale di poche decine di micrometri. 
    
    \subsection{Time Projection Chamber  (TPC)} \label{TPC}
    La \textit{Time Projection Chamber}  è il miglior rivelatore di ALICE per la ricostruzione delle tracce delle particelle prodotte nella collisione. Anche questo di forma cilindrica, si trova attorno all'ITS, ha un raggio interno di 85 $cm$ e esterno di 250 $cm$, mentre è lungo 500 $cm$ nella direzione del fascio di particelle. Il raggio interno è determinato dalla massima densità di tracce accettabile dalla TPC, quello esterno dalla lunghezza minima necessaria per una precisione del 10$\%$ sulla perdita specifica di energia $dE/dx$. 
     
     \begin{figure}[htbp]
        \centering
        \includegraphics[width=0.6\linewidth]{ALICE/ALICE-TPC-detector.png}
        \caption{Schema della Time Projection Chamber di ALICE}
        \label{fig:TPCcomplex}
    \end{figure}
    
    La camera del TPC è riempita con una miscela di gas, originariamente erano 90$\%$ di $Ne$ e 10$\%$ di $CO_2$, poi è stato aggiunto un 5$\%$ di $N_2$. Questa miscela di gas serve per trasportare gli elettroni di ionizzazione creati dal passaggio della particella da tracciare. Gli elettroni di ionizzazione vengono accelerati dal campo elettrico creato dagli elettrodi della TPC per arrivare alle placche esterne che permettono di registrare il segnale. \cite{Collaboration_2008_ALICE}
    \\La TPC è utilizzata per misurare il momento delle particelle. I magneti, all'interno dei quali si trovano i rivelatori di ALICE, curvano la traiettoria della particella carica, dalla curvatura della traccia ricostruita si ricava il valore del momento. Si utilizza la PID su un vasto range di momenti, questo è possibile grazie alla misura della perdita di energia specifica ( $dE/dx$ ), la carica e il momento della particella. Si confrontano i dati con i valori della Bethe-Bloch per identificare la particella considerata. Per valori del momento trasverso $p_T$ minori o uguali ad 1 $GeV/c$  la distinzione tra le particelle è chiara, per valori di $p_T$ più alti si usa la PID in modo statistico. 
    
    %AGGIUNGERE TOF!
    
    
    
    
    
    