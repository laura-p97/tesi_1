\chapter*{Introduzione}
\addcontentsline{toc}{chapter}{Introduzione}  
L'obbiettivo di questa tesi è quello di studiare la produzione di un mesone contenente un quark charm, il mesone $D^{*+}$,  in collisioni protone-protone, utilizzando i dati raccolti dall'esperimento ALICE (A Large Ion Collider Experiment) ad LHC (Large Hadron Collider) del Cern. Lo studio del mesone $D^{*+}$ viene ricostruito attraverso il decadimento $D^{*+} \rightarrow D^0 \pi^+$. Per l'analisi si utilizzeranno i dati relativi a 985~milioni di collisioni protone-protone ad un'energia nel centro di massa di 5~TeV.

Lo studio della produzione del mesone $D^{*+}$ presentato in questa tesi verrà condotto con un nuovo metodo di analisi multivariata e in particolare utilizzando un algoritmo chiamato Boosted Decision Tree (BDT). Per l'analisi in questa tesi si utilizza il TMVA \cite{TMVA} (Tool for MultiVariate Analysis), incluso nel framework di Root \cite{Root}. 

I risultati ottenuti verranno confrontati con i risultati pubblicati da ALICE ottenuti con un metodo di identificazione del mesone $D^{*+}$  basato su selezioni applicate alle variabili cinematiche e topologiche del decadimento \cite{dati_ALICE}. 
L'identificazione del mesone $D^{*+}$ è un tipico problema di selezione, in cui si cerca un segnale raro in un campione composto principalmente da eventi di fondo. L'utilizzo di algoritmi di apprendimento artificiale (machine learning) come il BDT può essere utile per risolvere problemi di selezione. Sar\`a quindi interessante valutare l'utilità di algoritmi di analisi multivariata per l'analisi dati di problemi di selezione di questo tipo ad ALICE.

La tesi \`e strutturata nel modo seguente:
    \begin{itemize}
        \item CAPITOLO 1: introduzione all'esperimento ALICE e ai principali rivelatori usati per l'identificazione del mesone $D^{*+}$;
        \item CAPITOLO 2: motivazioni per lo studio del mesone $D^{*+}$ in collisioni protone-protone;
        \item CAPITOLO 3: introduzione agli algoritmi di analisi multivariata e al funzionamento dell'algoritmo Boosted Decision Tree;
        \item CAPITOLO 4: descrizione della fase di training dell'algoritmo Boosted Decision Tree;
        \item CAPITOLO 5: descrizione dell'applicazione del Boosted Decision Tree ai dati di ALICE;
        \item CAPITOLO 6: conclusioni.
    \end{itemize}
